\documentclass[CheatSheet]{subfiles}

\newcommand\MD[1][]{M_{\mathrm{D}#1}}
\newcommand\MN[1][]{M_{N#1}}


\begin{document}
\summarystyle
\section{Neutrino}

(summary page)

\detailstyle
\clearpage
\subsection{Convention}
We extend the SM Yukawa \eqref{eq:SMyukawa} to include the neutrino mass terms:
\begin{align}
   \mathcal L\w{Y+\nu}&=
  \overline{U}\Yu H \PL Q
- \overline{D}\Yd H^\dagger \PL Q
+ \overline{N}\Yn H \PL L
- \overline{E}\Ye H^\dagger \PL L
- \frac{1}{2}\overline{N}\MN N^\cc
+ \text{h.c.}
\\&=
- \overline{Q^a}\Yu^\dagger \epsilon^{ab}H^{b*} \PR U
- \overline{Q^a}\Yd^\dagger H^{a} \PR D
- \overline{L^a}\Yn^\dagger \epsilon^{ab}H^{b*} \PR N
- \overline{L^a}\Ye^\dagger H^{a} \PR E
- \frac{1}{2}\overline{N}\MN N^\cc + \text{h.c.},
\end{align}
where $\MN$, a \emph{complex symmetric} Majorana mass matrix, may be absent.

As we explicitly describe in \cref{sec:PMNS}, we use the standard convention for the PMNS matrix \GRAY{(Pontecorvo--\japanese{牧}--\japanese{中川}--\japanese{坂田})}:
\begin{equation}
 \ket{\nu_\alpha^{\text{flavor}}} = [U\w{PMNS}]_{\alpha i}^*\ket{\nu_i^{\text{mass}}},
\qquad
 \nu^\text{flavor} = U\w{PMNS}\nu^{\text{mass}}, \qquad[\alpha=e,\mu,\tau,\mathrm{r}_1,\mathrm{r}_2,\dots]\label{eq:PMNS-defining}
\end{equation}
where the ``flavor basis'' is defined by the charged lepton mass basis for first three elements, while in most cases by the basis that diagonalizes Majorana mass term for the rest.


\paragraph{Dirac neutrino}
If $\MN$ is absent, neutrinos become Dirac fermions and the discussion goes parallel to the CKM matrix:
\begin{equation}
 \Yn = U_n Y_n\wdiag V_n^\dagger,\qquad
\{\nuL, \eL, \overline N, \overline E\}^{\text{mass basis}} = \{V_n^\dagger L^1, V_e^\dagger L^2, \overline N U_n, \overline E U_e\}
\end{equation}
and
\begin{align}
 \mathcal L&\supset\overline{L}\ii\gamma^\mu(-\ii g_2 W_\mu)\PL L
\supset
\frac{g_2}{\sqrt2}\left[
\overline{L^1} \slashed W^+\PL L^2
+\overline{L^2}\slashed W^-\PL L^1
\right]
\\&=
\frac{g_2}{\sqrt2}\left[
\overline{(L^1)^{\text{mass}}} V_n^\dagger V_e\slashed W^+\PL (L^2)^{\text{mass}}
+\overline{(L^2)^{\text{mass}}}V_e^\dagger V_n\slashed W^-\PL(L^1)^{\text{mass}}
\right]
\\&=
\frac{g_2}{\sqrt2}\left[
\overline{\nuL^{\text{mass}}} [U\w{PMNS}]^\dagger\slashed W^+\eL^{\text{mass}}
+\overline{\eL^{\text{mass}}}[U\w{PMNS}] \slashed W^-\nuL^{\text{mass}}
\right].
\end{align}
The mass eigenstate Dirac field is given by
\begin{equation}
 \nu^{\text{mass}} = \pmat{\nuL^{\text{mass}}\\N^{\text{mass}}} = 
\nuL^{\text{mass}}+N^{\text{mass}}=
V_n^\dagger L^1 + U_n^\dagger N
\qquad
 \Bigl(L^1 = \PL V_n \nu^{\text{mass}},~
 N = \PR U_n \nu^{\text{mass}}\Bigr).
\end{equation}
The PMNS matrix is given in \emph{the opposite manner} to the CKM matrix:
\begin{equation}
 U\w{PMNS}^{\text{Dirac}}=V_e^\dagger V_n.
\end{equation}
In general, $\Yn\in\mathbb C^{n\times3}$, $V_n\in\mathbb U_{\mathbb C}^3$, $U_n\in\mathbb U_{\mathbb C}^n$, and the Dirac-PMNS matrix is also a $3\times 3$ unitary matrix.
Models with $n<3$ yields $3-n$ massless left-handed neutrinos, while $n>3$ results in $n-3$ massless right-handed neutrinos.


\paragraph{Majorana neutrino}
Models with $\MN\neq0$ generate so-called Majorana neutrino masses.
If $n<3$, the model has $3-n$ massless neutrinos and $2n$ massive neutrinos, while all neutrinos are massive if $n\ge 3$.

Using Weyl spinor,
\begin{align}
& L^1 = \pmat{\nuL\\0},\quad
 N   = \pmat{0\\\bnR},\quad
 \MD := \frac{v}{\sqrt2}Y_n,\quad
 \MD\wdiag := U_n \MD Y_n^\dagger,\\
&\label{eq:MajoranaNuLag}
 \mathcal L\w{Y+\nu} 
\supset
 -\frac{v}{\sqrt2}\overline{N}Y_n \PL L^1 - \frac{1}{2}\overline{N}\MN N^\cc + \text{h.c.}
=
 -\frac{v}{\sqrt2}\nR Y_n \nuL - \frac{1}{2}\nR \MN\nR + \text{h.c.}\\
&\phantom{ \mathcal L\w{Y+\nu}}
=-\frac12\pmat{\nuL & \nR}\pmat{0 & \MD^\TT \\ \MD & \MN} \pmat{\nuL \\ \nR} + \text{h.c.} =: -\frac12\tilde\nu^\TT \tilde M\tilde \nu+\text{h.c.}
\end{align}
As $\tilde M$ is a complex symmetric $(3+n)\times(3+n)$ matrix, it can be AT-diagonalized:
\begin{equation}
  \tilde M = \tilde R \tilde M\wdiag \tilde R^\TT;
\qquad
-\mathcal L\w{Y+\nu}\supset
\frac12\tilde \nu^\TT \tilde M\tilde\nu
=\frac12(\tilde\nu^{\text{mass}})^\TT \tilde M\wdiag \tilde\nu^{\text{mass}};\qquad \tilde\nu^{\text{mass}} =\tilde R^\TT\tilde\nu
\end{equation}
and $\tilde M$ gives the neutrino masses.
The neutrino mixing is then given by
\begin{align}
 \pmat{\nuL\\\nR} = \tilde R^*\pmat{\nu_{1\TO3} \\ \nu_{4\TO}}
\end{align}
and find that the PMNS matrix is now extended to a $3\times n$ matrix $V_e^\dagger [\tilde R^*]\w{upper}$.

Usually the discussion should be start from the basis in which $Y_e$ and $M_N$ are positive diagonal with increasing diagonal elements (``charged lepton mass basis'' combined with ``Majorana mass basis'').
Then
\begin{equation}
\tilde M = \pmat{0 & \MD^\TT \\ \MD & \MN\wdiag},
\quad
\pmat{\nuL\\\nR}
=
\pmat{\begin{smallmatrix}\nu_e\\\nu_\mu\\\nu_\tau\end{smallmatrix}\\\nu_{\text{sterile};i}}
=\tilde R^*\pmat{\nu_{1\TO3}\\\nu_{4\TO}}.
\end{equation}
The PMNS matrix appears as a submatrices, which is no longer unitary:
\begin{equation}
\tilde R^*
=:
\pmat{ R_{11}^* & R_{12}^* \\ R_{21}^* & R_{22}^*}
=:
\pmat{U\w{PMNS}^{\text{Majorana}} & U_{\text{active-heavy}}\\
U_{\text{sterile-light}} & U_{\text{sterile-heavy}}
}.
\end{equation}


\paragraph{Connection of the above two formalism}
If we apply AT-diagonalization to the Dirac case (with $n=3$), we obtain
\begin{align}
\tilde M = \frac12\pmat{0 & (U_n\MD\wdiag V_n^\dagger)^\TT \\ U_n\MD\wdiag V_n^\dagger & 0},\quad
  \tilde R=\pmat{V_n^* & \ii V_n^* \\ U_n & -\ii U_n},\quad
  \tilde M\wdiag=\pmat{\MD\wdiag & 0 \\ 0 & \MD\wdiag}
\end{align}
and find that three pairs of degenerate Weyl fermions form three Dirac neutrinos. 


\paragraph{PMNS matrix}\label{sec:PMNS}
In the Dirac neutrino models the PMNS matrix is unitary, which we parameterize
\begin{equation}
\begin{split}
   U\w{PMNS}^{\text{Dirac}}&=V_e^\dagger V_n
 =\pmat{
 U_{e1} & U_{e2} & U_{e3}\\
 U_{\mu1} & U_{\mu2} & U_{\mu3}\\
 U_{\tau1} & U_{\tau2} & U_{\tau3}
 }
 =
 %
 \pmat{1&&\\&\co{23}&\si{23}\\&-\si{23}&\co{23}}
 \pmat{\co{13}&&\si{13}\ee^{-\ii\delta\w{CP}}\\&1&\\-\si{13}\ee^{\ii\delta\w{CP}}&&\co{13}}
 \pmat{\co{12}&\si{12}&\\-\si{12}&\co{12}&\\&&1}
 %
 \\&=
 \pmat{
 \co{12} \co{13} & \si{12} \co{13} & \si{13} \ee^{-\ii\delta\w{CP}}\\
 -\si{12} \co{23} - \co{12} \si{23} \si{13} \ee^{\ii\delta\w{CP}}& \co{12} \co{23} - \si{12}\si{23}\si{13} \ee^{\ii\delta\w{CP}}& \si{23}\co{13}\\
  \si{12}\si{23} - \co{12} \co{23} \si{13}\ee^{\ii\delta\w{CP}} & -\co{12}\si{23}-\si{12}\co{23}\si{13}\ee^{\ii\delta\w{CP}} & \co{23} \co{13}
},
\end{split}
\end{equation}
where $\theta_{ij}\in[0,\pi/2]$ and $\delta\w{CP}\in[0,2\pi]$ as shown in \cref{eq:GeneralUnitary33} and the five phases are removed as done in \cref{sec:SM-CKM}.

If the Majorana mass terms are present, $U\w{PMNS}^{\text{Majorana}}$, which is defined by \cref{eq:PMNS-defining}, is no longer unitary.
However, if $\MN\gg\MD$, it is approximately unitary:
\begin{equation}
   U\w{PMNS}^{\text{Majorana}}=
\pmat{
 U_{e1} & U_{e2} & U_{e3}\\
 U_{\mu1} & U_{\mu2} & U_{\mu3}\\
 U_{\tau1} & U_{\tau2} & U_{\tau3}
 }
\approx
U\w{PMNS}^{\text{Dirac}}
\pmat{\ee^{\ii\eta_1} \\ & \ee^{\ii\eta_2}\\&&1}.
\end{equation}
Here additional two phases are introduced because we can no longer rotate $\nR$; three among $(a,b,c,d,e)$ in \cref{eq:GeneralUnitary33} are removed by rotating $L_i$, and two remains.

Current experiments measure the value of the matrix and the above parameterization still works well (cf.~Ref.~\cite{NUFIT}).
Thus we hereafter identify the Dirac PMNS matrix as $U\w{PMNS}$:
\begin{align}
 &U\w{PMNS}:=U\w{PMNS}^{\text{Dirac}}=
 \pmat{
 \co{12} \co{13} & \si{12} \co{13} & \si{13} \ee^{-\ii\delta\w{CP}}\\
 -\si{12} \co{23} - \co{12} \si{23} \si{13} \ee^{\ii\delta\w{CP}}& \co{12} \co{23} - \si{12}\si{23}\si{13} \ee^{\ii\delta\w{CP}}& \si{23}\co{13}\\
  \si{12}\si{23} - \co{12} \co{23} \si{13}\ee^{\ii\delta\w{CP}} & -\co{12}\si{23}-\si{12}\co{23}\si{13}\ee^{\ii\delta\w{CP}} & \co{23} \co{13}
};\\
 &U\w{PMNS}^{\text{Majorana}} = U\w{PMNS}\pmat{\ee^{\ii\eta_1} \\ & \ee^{\ii\eta_2}\\&&1}
+\Order\left(\frac{\MD}{\MN}\right).
\end{align}
Accordingly, the components $U_{\alpha i}$ is in general slightly different from $(\alpha,i)$ component of $U\w{PMNS}$, e.g.,
\begin{equation*}
 U_{e1}\approx \co{12}\co{13}
\end{equation*}
with the exactness recovered in the Dirac case.

It should be noted that the discussion in \cref{sec:SM-FFdual} holds.
As far as the baryon number is conserved, we can remove the $\Theta_W W\tilde W$ term by quark rotation.
Hence, the above-discussed models have CP violation only in the CKM and (extended) PMNS matrices.




\paragraph{PDG and NuFIT convention} The convention agrees with PDG \cite[\S14]{PDG2020} and NuFIT~\cite[v5.0]{NUFIT}. Compared with PDG,\footnote{Sho thinks Eq.~(14.9) of PDG2020 lacks 1/2 in the right-most term.}
\begin{align*}
& \mathcal L
\supset
 -\CONV{\bar\nu_s} \CONV{M_D} \nu_L
 -\frac12\CONV{\bar\nu_s} \CONV{M_N} \CONV{\nu^c_s}
 -\overline{L^2}\CONV{M_l}\PR E\quad\text{(14.6)+(14.27)},\\
&
\Bigl\{\CONV{(V^{\nu})^\TT \spmat{0&M_D^\TT \\ M_D &M_N} V^\nu}\text{~or~}\CONV{V_R^{\nu\dagger}M_D V^\nu}\Bigr\}=\diag(m_i)\quad
\text{(Majorana; 14.9)+(Dirac; 14.15)},\\
&\nu_{L}=\PL\CONV{V^\nu}\nu^{\text{mass}}\quad\text{(14.14)\&(14.18)},\qquad
\CONV{V^{l\dagger}M_l V^l_R}=\diag(m_e,m_\mu,m_\tau)\quad\text{(14.31)}.
\end{align*}
These leads to $\CONV{V^\nu_R}=U_n$ and $\CONV{V^\nu}=V_n$ in Dirac case, $\CONV{M_D} = \Yn v/\sqrt2=\MD$, $\CONV{M_N}=M_N$, and $\CONV{V^\nu}=R^*$ in Majorana case, and $\CONV{V^{l}}=V_e$ and $\CONV{V_R^l}=U_e$ (note $\CONV{M_l}=Y_e^\dagger v/\sqrt{2}$).
So
\begin{equation*}
 \CONV{U_{ij}} = \text{(diagonal phases)}\times\CONV{V^{l\dagger}}\CONV{V^{\nu}}\times\text{(diagonal phases)}=\text{(phases)}V_e^\dagger \{V_\nu\text{~or~}R^*\}\text{(phases)}= U\w{PMNS}^{\text{(Dirac/Majorana)}}
\end{equation*}



\subsection{Casas--Ibarra parameterization}
\paragraph{General basis}
We start from the neutrino mass matrix
\begin{equation}
-\mathcal L\supset \frac12
 \pmat{\nuL & \nR} \pmat{0 & \MD^\TT \\ \MD & \MN} \pmat{\nuL \\ \nR}+\text{h.c.}
= \frac12\pmat{\nuL[i] & \nR[a]}
\pmat{0_{ij} & (\MD^\TT)_{ib} \\ \MD[aj] & \MN[ab]}
\pmat{\nuL[j] \\ \nR[b]}+\text{h.c.}
\end{equation}
The AT-factorization can be separated into two steps:
\begin{equation}
 \tilde M=\pmat{0 & \MD^\TT \\ \MD & \MN}
\longrightarrow
\pmat{M\w L & 0 \\ 0 & M\w H}
\stackrel{\text{ATF}}\longrightarrow
\pmat{M\w L\wdiag & 0 \\ 0 & M\w H\wdiag}=\tilde M\wdiag,
\end{equation}
where intermediate matrices $M\w L$ and $M\w H$ are complex symmetric.
This is expressed with unitary matrices $U$, $U_1$, and $U_2$:
\begin{equation}
 \tilde M =
U \pmat{M\w L & 0 \\ 0 & M\w H} U_1^\TT =
U \pmat{U_1 M\w L\wdiag U_1^\TT & 0 \\ 0 & U_2 M\w H\wdiag U_2^\TT} U^\TT,
\end{equation}
following the convention in \cref{eq:ATF}.
The first equality is calculated as an expansion in $\MD/\MN$ once we assume the see-saw mechanism:
\begin{align}
 U&\simeq \pmat{1 & \MD^\TT\MN^{-1} \\ -\MN^{-1}\MD^* & 1}
 - \frac12\pmat{\MD^\TT\MN^{-2}\MD^* & 0 \\ 0 & \MN^{-1}\MD^*\MD^\TT \MN^{-1}} + \cdots,\\
 M\w L&\simeq -\MD^\TT \MN^{-1} \MD+\cdots,\\
 M\w H&\simeq \MN + \frac12 (\MD\MD^\dagger\MN^{-1}+\MN^{-1}\MD^*\MD^\TT)+\cdots,
\end{align}
At the leading order,
\begin{equation*}
 U_1 M\w L\wdiag U_1^\TT = [U^\dagger \tilde M U^*]_{\text{upper left}} \approx -\MD^\TT \MN^{-1}\MD
\approx -\MD^\TT M\w H^{-1}\MD = -\MD^\TT (U_2 M\w H\wdiag U_2^\TT)^{-1}\MD,
\end{equation*}
or
\begin{equation*}
 -M\w L\wdiag \approx U_1^\dagger \MD^\TT U_2^*(M\w H\wdiag)^{-1}U_2^\dagger\MD U_1^*.
\end{equation*}
This is decomposed to
\begin{equation}
 [\ii M\w L\wdiag]^{1/2}[\ii M\w L\wdiag]^{1/2} = 
[(M\w H\wdiag)^{-1/2}U_2^\dagger\MD U_1^*]^\TT
[(M\w H\wdiag)^{-1/2}U_2^\dagger\MD U_1^*].
\end{equation}
This is the master equation for Casas-Ibarra parameterization~\cite{Casas:2001sr}.


\paragraph{``Standard'' parameterization}
Let us assume that we started from the above-discussed $(Y_e,\MN)$-diagonal basis.
Then, noting
\begin{equation}
 \tilde R=U\pmat{U_1\\&U_2}\approx \pmat{U_1 & \MD^\TT\MN^{-1}U_2 \\ -\MN^{-1}\MD^* U_1 & U_2},
\end{equation}
we can identify $U_1^*\approx U^{\text{Majorana}}\w{PMNS}$ and $U_2\approx 1$; the master equation now becomes
\begin{equation}
 [\ii M\w L\wdiag]^{1/2}[\ii M\w L\wdiag]^{1/2} =
[(M\w H\wdiag)^{-1/2}\MD U^{\text{Majorana}}\w{PMNS}]^\TT
[(M\w H\wdiag)^{-1/2}\MD U^{\text{Majorana}}\w{PMNS}].
\end{equation}
We will use this parameterization below.

\paragraph{Example: three right-handed neutrinos}
Let us assume all the neutrinos are massive thanks to three right-handed neutrinos. Then $M\w L\wdiag$ is invertible and
\begin{equation}
 R:=-\ii (M\w H\wdiag)^{-1/2}\MD  U^{\text{Majorana}}\w{PMNS}  (M\w L^\text{diag})^{-1/2}
\quad\Longrightarrow\quad R^\TT R=1.
\end{equation}
Conversely, with a matrix $R$ satisfying $R^\TT R=1$, the ``Yukawa matrix'' is given by
\begin{equation}
 \MD = \ii \sqrt{M\w H^\text{diag}}R\sqrt{M\w L^\text{diag}}(U^{\text{Majorana}}\w{PMNS} )^\dagger.
\end{equation}
Now we successfully parameterized $Y_n$ by a ``complex orthogonal'' matrix $R$.


Such $R$ is given by\footnote{For $w\in\mathbb C$, $\sin z_1=w$ and $\cos z_2=w$ always have solutions $z_{1,2}\in\mathbb C$. Meanwhile, $\tan z=w$ has no solution if and only if $w=\pm \ii$. Then, using this fact, one first expresses $R_{i3}$ components by $\zeta\w{A,B,C}=\pm1$ and $\theta\w{A,B}\in\mathbb C$, restricting $0\le\Re\theta\w{A,B}\le\pi/2$ ($\Leftrightarrow\Re{\sin{\theta}}\ge0\land\Re\cos\theta\ge0$), and then gets an expression of $R$ with three angles and six signs. Five signs are absorbed by enlarging $\Re\theta$ and one sign remains, which is $\zeta$.}
\begin{equation}
 R=\pmat{
 \co{12} \co{13} & \si{12} \co{13} & \si{13} \\
 -\zeta \si{12} \co{23}-\co{12} \si{23} \si{13} & \zeta \co{12} \co{23} -\si{12} \si{23} \si{13} & \si{23} \co{13} \\
  \zeta \si{12} \si{23}-\co{12} \co{23} \si{13} & -\zeta\co{12} \si{23} -\si{12} \co{23} \si{13} & \co{23} \co{13}
},
\end{equation}
where $\co{12}\equiv \cos{\theta_{12}}$ etc.\ and
\begin{equation}
 \zeta=\pm1; \qquad
(\theta_{12},\theta_{23},\theta_{13})\in\mathbb C,\quad
|\Re\theta_{12}|\le\pi, \quad |\Re\theta_{23}|\le\pi,  \quad |\Re\theta_{13}|\le\frac{\pi}{2}.
\end{equation}
This $R$ satisfies $RR^\TT=1$, which however is not general (as in the next example).

\paragraph{Example: two right-handed neutrinos}
For models with two right-handed neutrinos, one neutrino is massless and $M\w L\wdiag$ is not invertible. However the parameterization
\begin{equation}
 \MD = \ii \sqrt{M\w H^\text{diag}}R\sqrt{M\w L^\text{diag}}(U^{\text{Majorana}}\w{PMNS} )^\dagger
\end{equation}
works with\footnote{See, e.g., Ref.~\cite{Brdar:2019iem}. Sho also thanks Kai Schmitz for his note.}
\begin{align}
 R_{\text{normal hierarchy}}&=\pmat{0 & \cos z & \zeta\sin z \\ 0 & -\sin z & \zeta \cos z},&
 R_{\text{inverse hierarchy}}=\pmat{\cos z & \zeta\sin z &0 \\ -\sin z & \zeta \cos z & 0},
\end{align}
where $z\in\mathbb C$ and $\zeta=\pm1$.

\end{document}
