\documentclass[CheatSheet]{subfiles}

\begin{document}

\summarystyle
\setcounter{section}{-1}
\section{Notation and Convention}
This note uses the following notation and conventions unless otherwise stated.

\paragraph{Mathematics}
\begin{alignat}{2}
&\text{Pauli matrices}:&\quad&
 \sigma_0 = \spmat{1&0\\0&1},~
\sigma_x = \spmat{0&1\\1&0},~
\sigma_y = \spmat{0&-\ii\\\ii&0},~
\sigma_z = \spmat{1&0\\0&-1};\quad
\sigma^\mu=(1,\vc\sigma),~
\bar\sigma^\mu=(1,-\vc\sigma);
\notag\\
&&&
\sigma_+ = \tfrac12(\sigma_x+\ii\sigma_y) = \spmat{0&1\\0&0},\quad
\sigma_- = \tfrac12(\sigma_x-\ii\sigma_y) = \spmat{0&0\\1&0};
\\
&\text{Fourier transf.}:&&
 \tilde f(k) := \int\dd^4x\ \ee^{\ii kx}f(x); \qquad
        f(x) = \int\ddP[4]{k}\ \ee^{-\ii kx}\tilde f(k).
\end{alignat}


\paragraph{Relativity}
\begin{alignat}{2}
&\text{Minkowski metric}:&\quad&
\eta_{\mu\nu} = \eta^{\mu\nu}=\diag(+,-,-,-),
\qquad
\vep^{0123}_{0123}=\pm1.
\\
&\text{coordinates}:&&
x^\mu=(t,x,y,z),\quad
\partial_\mu=(\tfrac{\partial}{\partial t},\vc\nabla)
\\
&\text{electromagnetism\footnotemark}:&&
A^\mu=(\phi,\vc A),\quad
F_{\mu\nu}=\partial_\mu A_\nu-\partial_\nu A_\mu;\qquad
\epsilon^{\mu\nu\rho\sigma}\partial_\nu F_{\rho\sigma}=0,~
\partial_\mu  F^{\mu\nu}=e j^\nu.
\end{alignat}\footnotetext{
$\vc E=-\vc\nabla\phi+\dot{\vc A}$ and
$\vc B=\vc\nabla\!\times\!\vc A$, which lead to
$F_{\mu\nu}
=\spmat{0 & & \vc E & \\ & 0 & -B_3 & B_2 \\ -\vc E&B_3&0&-B_1\\&-B_2&B_1&0}_{\mu\nu}$,
$F_{\mu\nu} F^{\mu\nu}=2(\|\vc B\|^2-\|\vc E\|)$, 
 and the Maxwell equations
 $\vc\nabla\!\cdot\!\vc E = ej^0$,
 $\vc\nabla\!\cdot\!\vc B = 0$,
 $\vc\nabla\!\times\!\vc E+\dot{\vc B}=0$,
 $\vc\nabla\!\times\!\vc B-\dot{\vc E}=e\vc j.$
}

\paragraph{Spinors}
\begin{alignat}{2}
&\text{Gamma matrices}:&\quad&
\{\gamma^\mu,\gamma^\nu\}=2\eta^{\mu\nu},~
\gamma_5=\ii\gamma^0\gamma^1\gamma^2\gamma^3;\quad
\{\gamma^\mu,\gamma_5\}=0,~
\gamma^5\gamma^5=1.
\\
&\text{chiral notation}:&&
 \overline{\psi}=\psi^\dagger\gamma^0;~
 \gamma^\mu=\pmat{0&\sigma^\mu\\\bar\sigma^\mu&0},~
 \gamma_5=\pmat{-1&0\\0&1};~
 \PL = \frac{1-\gamma_5}{2},~
 \PR = \frac{1+\gamma_5}{2}.
\end{alignat}


\end{document}

