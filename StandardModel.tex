\documentclass[CheatSheet]{subfiles}
\begin{document}

\summarystyle
\section{Standard Model}
(summary page)

\newpage
\detailstyle

\subsection{Particle content and convention}
\subsection{Lagrangian}
\subsection{Higgs mechanism}

A general expression for composing a Dirac fermion from $\psi\w L(T\w{3L}, Y\w L)$ and $\psi\w R(T\w{3R}, Y\w R)$ is given by
\begin{align}
& \left(
  {g_2} \slashedW[_3] T\w{3L}
 +{g_Y} \slashed{B} Y\w{L}\right)\PL
 + \left(
  {g_2} \slashedW[_3] T\w{3R}
 +{g_Y} \slashed{B} Y\w{R}\right)\PR\\
&=\Bigl[
  \left(|e|\slashed A+g_Z\cw^2\slashed Z\right) T\w{3L}
 +\left(|e|\slashed A-g_Z\sw^2\slashed Z\right) Y\w {L}\Bigr]\PL
 + \text{(right)}\\
\begin{split}
&=
   \frac{T\w{3L}+T\w{3R}+Y\w L+Y\w R}{2}
  |e|\slashed A
  +\frac{T\w{3L}\cw^2 - Y\w {L}\sw^2 + T\w{3R}\cw^2 - Y\w {R}\sw^2}{2}
  g_Z\slashed Z\\
&\qquad
 +\frac{- T\w{3L} - Y\w {L} + T\w{3R} + Y\w {R}}2
 |e|\slashed A\gamma_5
 +\frac{ -\cw^2 T\w{3L} +\sw^2 Y\w {L} +\cw^2 T\w{3R} -\sw^2 Y\w {R}}{2}
 g_Z \slashed Z\gamma_5.
\end{split}
\end{align}
In the SM, $T\w{3L}+Y\w L=Y\w{R}=:Q$ and $T\w{3R}=0$ lead to
\begin{equation}
   Q |e|\slashed A
  +g_Z\slashed Z \left(T\w{3L}\PL - Q\sw^2\right).
\end{equation}
\subsection{Lagrangian in mass eigenstates}

\clearpage

\subsection{CKM matrix and Yukawa convention}\label{sec:SM-CKM}
We use the following convention for  the Yukawa interaction terms:
\begin{align}
  \mathcal L\w{Yukawa}&=
  \overline{U}\Yu H \PL Q
- \overline{D}\Yd H^\dagger \PL Q
- \overline{E}\Ye H^\dagger \PL L + \text{h.c.}\label{eq:SMyukawa}
\\&=
  \overline{U_i}\Yu[ij] \epsilon^{ab}H^a \PL Q_j^b
- \overline{D_i}\Yd[ij] H^{a*} \PL Q_j^a
- \overline{E_i}\Ye[ij] H^{a*} \PL L_j^a + \text{h.c.}
\\&=
- \overline{Q^a}\Yu^\dagger \epsilon^{ab}H^{b*} \PR U
- \overline{Q^a}\Yd^\dagger H^{a} \PR D
- \overline{L^a}\Ye^\dagger H^{a} \PR E + \text{h.c.},
\end{align}
where the last equality uses $(\overline{\psi_A}\PL\psi_B)^*=\overline{\psi_B}\PR\psi_A$.

These terms are diagonalized by the singular value decomposition $Y=UY\wdiag V^\dagger$ (see \cref{app:diagonalization}):
\begin{align}
\mathcal L\w{Yukawa}&=
  \epsilon^{ab}\overline{U}U_u\Yu\wdiag H^a \PL V_u^\dagger Q^b
- \overline{D}U_d\Yd\wdiag H^{a*} \PR V_d^\dagger Q^a
- \overline{E}U_e\Ye\wdiag H^{a*} \PR V_e^\dagger L^a + \text{h.c.}
\\&\leadsto
  -\frac{v}{\sqrt2}\overline{U}U_u\Yu\wdiag V_u^\dagger\PL Q^1
  -\frac{v}{\sqrt2}\overline{D}U_d\Yd\wdiag V_d^\dagger\PL Q^2
  -\frac{v}{\sqrt2}\overline{E}U_e\Ye\wdiag V_e^\dagger\PL L^2 + \text{h.c.}
\end{align}
under the EWSB with $v\simeq246\GeV$. Mass eigenstates are
\begin{equation}
 \{Q^1, Q^2, L, \overline U, \overline D, \overline E\}^\text{mass basis}
=
\{V_u^\dagger Q^1, V_d^\dagger Q^2, V_e^\dagger L,
  \overline UU_u, \overline DU_d, \overline EU_e
\}
\end{equation}
and, since $Q^1$ and $Q^2$ are rotated by different matrices, the weak interaction receives flavor violation.amended as
\begin{align}
 \mathcal L&\supset\overline{Q}\ii\gamma^\mu(-\ii g_2 W_\mu)\PL Q
\supset
\frac{g_2}{\sqrt2}\left[
\overline{Q^1} \slashed W^+\PL Q^2
+\overline{Q^2}\slashed W^-\PL Q^1
\right]
\\&=
\frac{g_2}{\sqrt2}\left[
\overline{(Q^1)^{\text{mass}}} V_u^\dagger\slashed W^+\PL V_d (Q^2)^{\text{mass}}
+\overline{(Q^2)^{\text{mass}}}V_d^\dagger\slashed W^-\PL V_u (Q^1)^{\text{mass}}
\right]
\\&=
\frac{g_2}{\sqrt2}\left[
\overline{(Q^1)^{\text{mass}}} V\w{CKM}\slashed W^+\PL (Q^2)^{\text{mass}}
+\overline{(Q^2)^{\text{mass}}}V\w{CKM}^\dagger \slashed W^-\PL(Q^1)^{\text{mass}}
\right],
\end{align}
where the CKM matrix are defined with positive angles:
\begin{align}
  V\w{CKM}&=V_u^\dagger V_d
 =\pmat{
 V_{ud} & V_{us} & V_{ub}\\
 V_{cd} & V_{cs} & V_{cb}\\
 V_{td} & V_{ts} & V_{tb}
 }
 =
%
 \pmat{1&&\\&\co{23}&\si{23}\\&-\si{23}&\co{23}}
 \pmat{\co{13}&&\si{13}\ee^{-\ii\delta}\\&1&\\-\si{13}\ee^{\ii\delta}&&\co{13}}
 \pmat{\co{12}&\si{12}&\\-\si{12}&\co{12}&\\&&1}
%
 \\&=
 \pmat{
 \co{12} \co{13} & \si{12} \co{13} & \si{13} \ee^{-\ii\delta}\\
 -\si{12} \co{23} - \co{12} \si{23} \si{13} \ee^{\ii\delta}& \co{12} \co{23} - \si{12}\si{23}\si{13} \ee^{\ii\delta}& \si{23}\co{13}\\
  \si{12}\si{23} - \co{12} \co{23} \si{13}\ee^{\ii\delta} & -\co{12}\si{23}-\si{12}\co{23}\si{13}\ee^{\ii\delta} & \co{23} \co{13}
 }\qquad[\si{ij}>0,~\co{ij}>0].
\end{align}
Here in the Standard Model, five phases in a unitary matrix \eqref{eq:GeneralUnitary33} are removed by rotating fermion phases.


\paragraph{PDG convention} \cite[\S12]{PDG2018}\cite[\S12]{PDG2020}
\begin{equation}
 \mathcal L \supset
 -\C{Y^d_{ij}}\overline{Q_{Li}^I}\C{\phi} d_{Rj}^I
 -\C{Y^u_{ij}}\overline{Q_{Li}^I}\C{\epsilon\phi^*}u_{Rj}^I,
\quad Y\wdiag = \C{V_L Y V_R^\dagger},
\quad \C{V\w{CKM}} = \C{V^u_LV_L^{d\dagger}}.
\end{equation}
So, $\C{Y^u}=\Yu^\dagger$, $\C{Y^d}=\Yd^\dagger$;
$Y\wdiag = \C{V_RY^\dagger V_L^\dagger} = \C{V_R}Y\C{V_L^\dagger}$ leads $\C{V_L}=V^\dagger$, and the CKM matrix (and components) is in the same convention: $\C{V\w{CKM}}=V_u^\dagger V_d=V\w{CKM}$.


\paragraph{SLHA2 convention} \cite{SLHA2}
\begin{align}
 W &\supset \epsilon_{ab}\left[
(\C{Y_E})_{ij}H_1^a L_i^b \bar E_j
+(\C{Y_D})_{ij}H_1^a Q_i^b \bar D_j
+(\C{Y_U})_{ij}H_2^b Q_i^a \bar U_j
\right];
\\
\mathcal L&\supset
- \epsilon_{ab}\left[
(\C{Y_E})_{ij}H_1^a \psi_{Li}^b \psi_{\bar Ej}
+(\C{Y_D})_{ij}H_1^a \psi_{Qi}^b \bar \psi_{\bar Dj}
+(\C{Y_U})_{ij}H_2^b \psi_{Qi}^a \bar \psi_{\bar Uj}
\right]
\\&\leadsto
- \left[
\psi_{\bar E}\vd \C{Y_E^\TT}\psi_L^2
+\psi_{\bar D}\vd\C{Y_D^\TT}\psi_Q^2
+\psi_{\bar U}\vu\C{Y_U^\TT}\psi_Q^1
\right];
\quad
Y\wdiag = \C{U^\dagger Y^\TT V},
\quad
\C{V\w{CKM}} = \C{V_u^\dagger V_d}.
\end{align}
Hence, $\C{Y_E}=\Ye^\TT$, $\C{Y_D}=\Yd^\TT$, $\C{Y_U}=\Yu^\TT$;
$Y\wdiag=\C{U^\dagger}Y\C{V}$, $\C{V}=V$ and $\C{V\w{CKM}}=V\w{CKM}$.

\paragraph{Wolfenstein parameterization}
The CKM matrix is precisely written in terms of $\lambda$, $A$, and $\bar\rho+\ii\bar\eta$.
\begin{equation}
  \lambda := \si{12} = \frac{|V_{us}|}{\sqrt{|V_{ud}|^2+|V_{us}|^2}},
\quad
  A:=\frac{\si{23}}{\lambda^2}=\lambda^{-1}\left|\frac{V_{cb}}{V_{us}}\right|,
\quad
\bar\rho+\ii\bar\eta:=\frac{-V_{ud}V^*_{ub}}{V_{cd}V^*_{cb}}.
\end{equation}
They are independent of the phase convention and used for SLHA2 input, i.e., \texttt{VCKMIN} should contain $(\lambda, A, \bar\rho, \bar\eta)$.

Also, $\bar\rho+\ii\eta$ is approximately written by
\begin{equation}
 R=\rho+\ii\eta:=\frac{s_{13}\ee^{\ii\delta}}{A\lambda^3}
=\frac{V^*_{ub}}{A\lambda^3}\frac{V_{ud}}{|V_{ud}|}
= \frac{(\bar\rho+\ii\bar\eta)\sqrt{1-A^2\lambda^4}}
        {\sqrt{1-\lambda^2}\left[1-A^2\lambda^4(\bar\rho+\ii\bar\eta)\right]}
= \left(\bar\rho+\ii\bar\eta\right)\left(1+\frac{\lambda^2}{2}+\Order(\lambda^4)\right),
\end{equation}
with which
\begin{equation}
 V\w{CKM}
=\pmat{
 1-\lambda^2/2 & \lambda       & A\lambda^3R^*\\
 -\lambda      & 1-\lambda^2/2 & A\lambda^2\\
 A\lambda^3(1-R) & -A\lambda^2 & 1
}\ee^{\ii\Theta}
+\pmat{
\Order(\lambda^4) & \Order(\lambda^7) & 0\\
\Order(\lambda^5) & \Order(\lambda^4) & \Order(\lambda^8)\\
\Order(\lambda^5) & \Order(\lambda^4) & \Order(\lambda^4)
}.
\end{equation}

\subsection{General Higgs doublet and Nambu--Goldstone bosons}
In linear parameterization,
\begin{align}
&H=\frac1{\sqrt2}\pmat{\ii\sqrt2\phi^+ \\ v+h+\ii\phi_3},
\qquad
 D_\mu H =
\pmat{
 \ii\partial_\mu\phi^+
 -\frac{\ii g_2}{2}(v+h+\ii\phi_3)W^+_\mu
 +\left(|e|A_\mu+\frac{\cw^2-\sw^2}{2}g_ZZ_\mu\right)\phi^+
\\
 \partial_\mu(h+\ii\phi_3)/\sqrt2
+ \frac{\ii g_Z}{2} Z_\mu(v+h+\ii\phi_3)/\sqrt2
+ g_2 W^-_\mu\phi^+/\sqrt2
};
\\
\begin{split}
 &|D_\mu H|^2=
 \frac{(\partial_\mu h)^2 + (\partial_\mu \phi_3)^2}{2} + \partial_\mu \phi^+\partial^\mu\phi^-
 +
 \frac{(v+h)^2}{8}(2g_2^2 W^{+\mu}W^-_\mu+g_Z^2Z^\mu Z_\mu)
 \\&\qquad
 + \frac{\partial^\mu h}{2}\left[g_2 W^+_\mu\phi^-+g_2W^-_\mu\phi^+-g_Z Z_\mu\phi_3\right]
 + \frac{\partial^\mu\phi_3}{2}
         \left[g_Z(v+h)Z_\mu + \ii g_2(W^+_\mu\phi^--W^-_\mu\phi^+)\right]
 \\&\qquad
+ \left\{
 \frac{\partial^\mu\phi^+}{2}\left[
 -g_2(v+h-\ii\phi_3)W^-_\mu + (2|e|A_\mu+(\cw^2-\sw^2)g_Z Z_\mu)\ii\phi^-
 \right]
 +\text{H.c.}\right\}
 \\&\qquad
 +\frac{\ii g_2(v+h)}{2}(|e|A^\mu-g_Z\sw^2 Z^\mu)(W^-_\mu\phi^+ - W^+_\mu\phi^-)
 +  \frac{g_2\phi_3}{2}(|e|A^\mu-g_Z\sw^2 Z^\mu)(W^-_\mu\phi^+ + W^+_\mu\phi^-)
 \\&\qquad
 +\frac{\phi_3^2}{8}\left(2g_2^2 W^{+\mu}W^-_\mu+g_Z^2Z^\mu Z_\mu\right)
 +\frac{\phi^+\phi^-}{4}
 \left[
 2g_2^2 W^{+\mu}W^-_\mu + \left(2|e|A_\mu+g_Z(\cw^2-\sw^2)Z_\mu\right)^2
 \right];
\end{split}
\\&V=\lambda|H|^4-\mu^2|H|^2=
\frac{\lambda}{4}h^4+\lambda v h^3+\frac{2\lambda v^2}{2}h^2-\frac{\lambda}{4}v^4
+\frac{\lambda}{4}(2\phi^+\phi^-+\phi_3^2)^2 + \frac{\lambda}{2}(h^2+2vh)(2\phi^+\phi^-+\phi_3^2),
\end{align}
where $v=\mu/\sqrt\lambda\sim246\GeV$, $\lambda\sim0.13$, and $\mu\sim89\GeV$.
In exponential parameterization,
\begin{align}
&H=\frac1{\sqrt2}\exp\left(\frac{\ii}{v}\sigma_i\varphi_i\right)\pmat{0\\v+h},
\\&
 D_\mu H
= \frac1{\sqrt2}\ee^{\ii\sigma_i\varphi_i/v}
\left[\ii\sigma_i\partial_\mu\varphi_i\pmat{0\\1+h/v} + \pmat{0\\\partial_\mu h}\right]
+\frac1{\sqrt2}\frac{-\ii}{2}(g_2\sigma_i W_{i\mu}+g_Y B_\mu)
\ee^{\ii\sigma_i\varphi_i/v}\pmat{0\\v+h}
\\&\phantom{D_\mu H}
= \frac1{\sqrt2}\ee^{\ii\sigma_i\varphi_i/v}
\left[
\pmat{0\\\partial_\mu h}
+
\ii
\left(
\sigma_i\partial_\mu\varphi_i
-\frac{g_2v}{2}
\ee^{-\ii\sigma_j\varphi_j/v}
\sigma_i
\ee^{\ii\sigma_k\varphi_k/v}
W_{i\mu}-
\frac{g_Y v}{2}B_\mu
\right)
\pmat{0\\1+h/v}\right],
\\&V=\lambda|H|^4-\mu^2|H|^2=
\frac{\lambda}{4}h^4+\lambda v h^3+\frac{2\lambda v^2}{2}h^2-\frac{\lambda}{4}v^4.
\end{align}
These expressions have gauge degeneracy (i.e., without gauge-fixing terms and  ghost terms) and thus not ready for calculations.
If we choose the unitarity gauge, $\phi_i(x)=0$,
\begin{equation}
\mathcal L_H= |D_\mu H|^2-V(H)=\frac12(\partial_\mu h)^2-\frac{2\lambda v^2}{2}h^2
-\frac{\lambda}{4}h^4-\lambda v h^3
 + \frac{(v+h)^2}{8}(2g_2^2 W^{+\mu}W^-_\mu+g_Z^2Z^\mu Z_\mu)
+\frac{\lambda}{4}v^4.
\end{equation}


\subsection[CP-violating FFdual terms]{CP-violating $F\tilde F$ terms}
\label{sec:SM-FFdual}
The Standard Model contains CP-violating terms
\begin{equation}
 \mathcal L_{\text{gauge},\cancel{CP}}
=
\frac{g_3^2\Theta_g}{64\pi^2}\epsilon^{\mu\nu\rho\sigma}G^a_{\mu\nu}G^a_{\rho\sigma}
+\frac{g_2^2\Theta_W}{64\pi^2}\epsilon^{\mu\nu\rho\sigma}W^a_{\mu\nu}W^a_{\rho\sigma}
+\frac{g_Y^2\Theta_B}{64\pi^2}\epsilon^{\mu\nu\rho\sigma}B_{\mu\nu}B_{\rho\sigma}.
\end{equation}
We here discuss we can ignore $\Theta_W$ and $\Theta_B$, while $\Theta_g$ causes the strong $CP$ roblem.

One should first note that the value of $\Theta_i$ depends on the basis of the chiral fermions: in \cref{sec:SM-CKM} fermions are redefined by rotations. These rotations generate these terms and the angles are modified.
It is then found that $\Theta_W$ can be rotated away.
Let us see this explicitly, starting from the mass basis, i.e., $Y_{u,d,e}$ are positive diagonal and $\gSU(2)$ interactions are amended by $V\w{CKM}$.
As we do not introduce phases in, e.g., $W$--$u$--$d$ interaction and fermion mass matrix, the possible rotation is limited to
\begin{equation}
 (Q,U,D)\to\ee^{\ii\theta}(Q,U,D), \qquad
 (L_i, E_i)\to\ee^{\ii\theta_i}(L_i,E_i).
\end{equation}
These rotations affect the CP-violating terms (Cf.~Fujikawa method):\footnote{%
Fail-safe memo:
chiral transformation $\psi\to\exp[\ii\gamma_5\alpha(x)]\psi$ generates
$\Delta\mathcal L = -(g^2/16\pi^2)\Tr[\alpha F\tilde F]$ (cf.~Weinberg II Eq.(2.2.24) but the overall sign may differ). For a constant (and non-matrix) $\alpha$, 
$\Delta\mathcal L = -(\alpha g^2/32\pi^2)F^a\tilde F^a$.
Also, the absence of gauge anomaly means the corresponding gauge transformations do not induce additional $\Theta$-terms.
}
\begin{align}
 \Delta \Theta_W &\propto 9\theta_Q + \sum\theta_{L_i} = 9\theta + \sum\theta_i&
 \Delta \Theta_B &\propto \frac12 \theta_Q + \frac32 \theta_L - (4\theta_U+\theta_D+3\sum\theta_{E_i}) = -\frac92 \theta - \frac12 \sum\theta_i,
\end{align}
which means either $\Theta_W$ or $\Theta_B$ can be rotated away.
As we discuss below, it is convenient to set $\Theta_W=0$ and $\Theta_B\neq 0$.

Meanwhile, because $\Delta\Theta_g=0$, we cannot remove $\Theta_g$.\footnote{If, e.g., $u$ were massless, we can take $\theta_{\uR}\neq\theta_Q$ and rotate $\Theta_g$ away.}
We define $\Theta\w{QCD}:=(\text{$\Theta_g$ in the mass basis})$, which induces $CP$-violation in the strong sector.
However, such $CP$-violation is not observed yet; this contradiction is called strong $CP$ problem.

The form $\epsilon^{\mu\nu\rho\sigma}F^a_{\mu\nu}F^a_{\mu\nu}$ is a total derivative and the effect is pushed away to the surface.\footnote{%
  Sho thanks to Kyohei Mukaida and Teppei Kitahara for a very useful discussion.
}
As discussed in \cite[\S23]{WeinbergQFT2}, the $\gU(1)_Y$ surface term does not do anything (in the simple spacetime) but the $\gSU(N)$ surface term corresponds to topologically non-trivial configuration of the gauge fields, labeled by a winding number $\nu$.
Such different configuration should be summed up in, e.g., the path integral formalism, and observed as the instanton effect (``sphaleron'' for $\gSU(2)_W$).
If $\Theta_W\neq 0$, the processes $\nu\to\nu\pm1$ would have different rate and $CP$ would be violated in the processes.
As $\Theta_B$ is not related to such process, we take $\Theta_W=0$ and, though $\Theta_B\neq 0$, do not further consider $\Theta_B$.






\end{document}

