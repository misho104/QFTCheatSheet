\documentclass[CheatSheet]{subfiles}
\begin{document}

\detailstyle
\section{Mathematics}
\subsection{Matrix exponential}
Excerpted from \S2 and \S5 of Hall 2015 \cite{Hall2015}:
\begin{alignat}{2}
 &\ee^X := \sum_{m=0}^\infty \frac{X^m}{m!} \text{~~(converges for any $X$)},
\quad&
 &\log X := \sum_{m=1}^\infty (-1)^{m+1}\frac{(A-1)^m}{m} \text{~~(conv.~if $\|A-I\|<1$)}.
\end{alignat}
\begin{alignat}{2}
 &\ee^{\log A} = A \text{~~(if $\|A-I\|<1$)},
\quad&
 &\log \ee^X = X \text{~and~} \|\ee^X-1\| < 1 \text{~~(if $\|X\|<\log2$).}
\end{alignat}
\begin{equation}
 \text{Hilbert-Schmidt norm}:~\|X\|^2 := \sum_{i,j}|X_{ij}|^2 = {\Tr X^\dagger X}.
\end{equation}
Properties:
\begin{alignat*}{4}
 \ee^{(X^\TT)} &= (\ee^X)^\TT,
&\qquad
 \ee^{(X^*)} &= (\ee^X)^*,
&\qquad
 (\ee^X)^{-1} &= \ee^{-X},
&\qquad
 \ee^{YXY^{-1}} &= Y\ee^X Y^{-1},
\end{alignat*}
\begin{alignat*}{2}
 \det \exp X &= \exp\Tr X,
&\qquad
 \ee^{(\alpha+\beta)X} &= \ee^{\alpha X}\ee^{\beta X} \text{~for $\alpha, \beta\in \mathbb C$};
\end{alignat*}
Baker-Campbell-Hausdorff:
\begin{align}
  \ee^X Y \ee^{-X}
&= Y + [X, Y] + \frac1{2!}[X, [X, Y]] + \frac1{3!}[X, [X, [X, Y]]] + \cdots = \ee^{[X,]}Y;
\\
  \ee^X \ee^Y \ee^{-X}
&= \sum_{n=0}^\infty\frac{1}{n!}(\ee^X Y \ee^{-X})^n
 = \exp\left(\ee^{[X,]}Y\right);
\\
  \log(\ee^X \ee^Y)
&= X + \int_0^1\!\dd t\, g(\ee^{[X,}\ee^{t[Y,})Y
\qquad\Bigl[
  g(z) = \frac{\log z}{1-z^{-1}} = 1-\sum_{n=1}^{\infty}\frac{(1-z)^n}{n(n+1)};
\quad
g(\ee^y)=\sum_{n=0}^{\infty}\frac{B_ny^n}{n!}
\Bigr]\\
&= X + Y + \frac12[X,Y] + \frac1{12}[X, [X,Y]] - \frac1{12}[Y, [X,Y]] + \cdots
\quad\text{(Baker-Campbell-Hausdorff).}
\end{align}
\begin{align}
   \log(\ee^X\ee^Y)
&=\sum_{k=1}^\infty \frac{(-1)^{k-1}}{k}\left(\sum_{m,n=0}^{\infty}\frac{X^mY^n}{m!n!}-1\right)^k
&=\sum_{k=1}^\infty
   \sum_{m_1+n_1>0}\cdots\sum_{m_k+n_k>0}\frac{(-1)^{k-1}}{k}
\frac{X^{m_1}Y^{n_1}\cdots X^{m_k}Y^{n_k}}{m_1!n_1!\cdots m_k!n_k!}
\end{align}
\begin{equation}
   \log(\ee^X\ee^Y)=\sum_{k=1}^{\infty}\sum_{m_1+n_1>0}\cdots\sum_{m_k+n_k>0}
\frac{(-1)^{k-1}}{k\sum_{i=1}^{k}(m_i+n_i)}
\frac{
\Bigl([X,\Bigr)^{m_1}
\Bigl([Y,\Bigr)^{n_1}\cdots
\Bigl([X,\Bigr)^{m_k}
\Bigl([Y,\Bigr)^{n_k}]\cdots]
}{m_1!n_1!\cdots m_k!n_k!}
\end{equation}
\hfill with $[X]$ being $X$.
\\
Derivative:
\begin{alignat}{1}
 &\frac{\dd}{\dd t}\ee^{tX} = X\ee^{t X} = \ee^{tX} X\\
 &\ee^{-X(t)}\left(\frac{\dd}{\dd t}\ee^{X(t)}\right) =
    \frac{I-\ee^{-\ad_X}}{\ad_X}\left(\frac{\dd X}{\dd t}\right)
  = X' + \frac{[-X,X']}{2!} + \frac{[-X,[-X,X']]}{3!} + \cdots
\\
 &\left(\frac{\dd}{\dd t}\ee^{X(t)}\right)\ee^{-X(t)}
= X' + \frac{[X,X']}{2!} + \frac{[X,[X,X']]}{3!} + \cdots
\end{alignat}
\hfill where $X'=\dd X/\dd t$ and $\ad_X(Y)=[X,Y]$ is the adjoint action of a Lie algebra. Thus, explicitly,
\begin{alignat}{1}
\frac{\dd}{\dd t}\ee^{aX(t)}
  = \ee^{aX}\left\{\sum_{n=0}^\infty\frac{a^{n+1}}{(n+1)!}\Bigl([-X,\Bigr)^n X']\right\}
  = \left\{\sum_{n=0}^\infty\frac{a^{n+1}}{(n+1)!}\Bigl([X,\Bigr)^n X']\right\}\ee^{aX}
\end{alignat}

\vspace{1em}

\noindent
Component:\quad If matrices $t^a$ satisfies $[t^a,t^b]=\ii f^{abc} t^c$ with totally-antisymmetric $f^{abc}\in\mathbb R$,
\begin{equation}
  \left[\ee^{\theta^at^a} t_b \ee^{-\theta^ct^c}\right]_{ij}
=  \left[\ee^{\theta^a[t^a,}t_b\right]_{ij}
= \left[\ee^{\ii \theta^af^{a}}\right]^{bc} t^c_{ij}
\end{equation}
holds for $\theta^a\in\mathbb C$, where $[f^a]_{bc}=f^{abc}$. \TODO{needs verification, generalization/restriction, and a nice proof or reference.}


\subsection{General unitary matrix}
\begin{equation}
 U_2 =
\pmat{1&0\\0&\ee^{\ii\alpha}}\pmat{\co\theta&\si\theta\\-\si\theta&\co\theta}\pmat{\ee^{\ii\beta}&0\\0&\ee^{\ii\gamma}}
=
\pmat{\phantom{-}
\co\theta\ee^{\ii\beta} & \si\theta\ee^{\ii\gamma}\\
-\si\theta\ee^{\ii(\alpha+\beta)} & \co\theta\ee^{\ii(\alpha+\gamma)}
},\qquad
0\le\theta\le\frac{\pi}2,\quad \alpha,\beta,\gamma\in\mathbb R;
\end{equation}
\begin{align}
 U_3 &=
\pmat{1&&\\&\ee^{\ii a}&\\&&\ee^{\ii b}}
 \pmat{1&&\\&\co{23}&\si{23}\\&-\si{23}&\co{23}}
 \pmat{\co{13}&&\si{13}\ee^{-\ii\delta}\\&1&\\-\si{13}\ee^{\ii\delta}&&\co{13}}
 \pmat{\co{12}&\si{12}&\\-\si{12}&\co{12}&\\&&1}
\pmat{\ee^{\ii c}&&\\&\ee^{\ii d}&\\&&\ee^{\ii e}}
\\&=
\pmat{1&&\\&\ee^{\ii a}&\\&&\ee^{\ii b}}
 \pmat{
 \co{12} \co{13} & \si{12} \co{13} & \si{13} \ee^{-\ii\delta}\\
 -\si{12} \co{23} - \co{12} \si{23} \si{13} \ee^{\ii\delta}& \co{12} \co{23} - \si{12}\si{23}\si{13} \ee^{\ii\delta}& \si{23}\co{13}\\
  \si{12}\si{23} - \co{12} \co{23} \si{13}\ee^{\ii\delta} & -\co{12}\si{23}-\si{12}\co{23}\si{13}\ee^{\ii\delta} & \co{23} \co{13}}
\pmat{\ee^{\ii c}&&\\&\ee^{\ii d}&\\&&\ee^{\ii e}}
\label{eq:GeneralUnitary33}
\end{align}
with $0\le \theta_{ij}\le\pi/2$ and $a,b,c,d,e,\delta\in\mathbb R$ (see, e.g., Ref.~\cite{Rasin:1997pn}).

\begin{minted}{wolfram}
U3 = Dot[
  DiagonalMatrix[Exp[I {0, a, b}]],
  RotationMatrix[\[Theta]23, {-1, 0, 0}],
  DiagonalMatrix[Exp[I {0,0,+\[Delta]}]],
  RotationMatrix[\[Theta]13, {0, 1, 0}],
  DiagonalMatrix[Exp[I {0,0,-\[Delta]}]],
  RotationMatrix[\[Theta]12, {0, 0, -1}],
  DiagonalMatrix[Exp[I {c, d, e}]]
]
\end{minted}


\subsection{Matrix diagonalization}\label{app:diagonalization}
In this section, $\mathbb K=\mathbb{R}$ or $\mathbb{C}$ and $\mathbb U_{\mathbb K}^{n}\subset \mathbb K^{n\times n}$ is the set of the unitary matrices.

\paragraph{Diagonalization}
A matrix $M\in\mathbb K^{n\times n}$ is called diagonalizable if $\exists P$ and $\exists D$ s.t.
\begin{equation}
 M=PDP^{-1};\qquad
 P\in\mathbb{K}^{n\times n},\quad
 D:\text{diagonal matrix}~(D_{ii}\in\mathbb{K}).
\end{equation}
In particular,
\begin{equation}
 \text{$M$ is normal} \stackrel{\text{def}}\iff M^\dagger M = M M^\dagger \iff
 \exists P\in\mathbb{U}_{\mathbb K}^{n} \text{~s.t.~} M=PDP^{-1}.
\end{equation}


\paragraph{Singular value decomposition}
Any $M\in\mathbb{K}^{m\times n}$ can be singular-value decomposed as
\begin{equation}
 M=UDV^\dagger;\qquad
 U\in\mathbb{U}_{\mathbb K}^m,\quad
 V\in\mathbb{U}_{\mathbb K}^n,\quad
 D: \text{non-negative real diagonal matrix}~(D_{ii}\ge0).
\end{equation}
Here, the matrix $U$ ($V$) diagonalizes $MM^\dagger$ ($M^\dagger M$) and $(D_{ii})^2$ are the eigenvalues of $MM^\dagger$ (and $M^\dagger M$).

The calculation on Mathematica is straightforward for this convention:
\begin{minted}{wolfram}
{u, d, v} = SingularValueDecomposition[M]
\end{minted}




\paragraph{Autonne-Takagi factorization}
If $M\in\mathbb{C}^{n\times n}$ is symmetric, it can be decomposed as
\begin{equation}
 M=RDR^\TT;\qquad
 R\in\mathbb{U}_{\mathbb C}^n,\quad
 D: \text{non-negative real diagonal matrix}~(D_{ii}\ge0).
 \label{eq:ATF}
\end{equation}
Real symmetric matrices are normal and thus do not need this factorization; we can apply the above ``diagonalization'' method.

Sample Mathematica code to calculate $\{D,R\}$ (with ordering, if specified) is:
\begin{minted}{wolfram}
AutonneTakagi[M_, order_ : None] := Module[{v0, v, p, ord, R, D},
  ord = If[order === None, Range[Length[M]], order];
  v0 = Eigenvectors[Conjugate[M].M];
  v = Eigenvectors[v0.M.Transpose[v0]].v0; (*resolve degenerate eigenvalues*)
  p = DiagonalMatrix[If[Abs[#] > 0, (#/Abs[#])^(-1/2), 1] & /@ Diagonal[v.M.Transpose[v]]];
  R = ConjugateTranspose[Reverse[p.v][[ord]] // Orthogonalize];
  D = ConjugateTranspose[R].M.Conjugate[R];
  {D, R}];
\end{minted}


\subsection{Group theory}\label{sec:group-theory}

\paragraph{Lie Groups}
This section is mainly based on Ref.~\cite{Hall2015}. Also, $\KK=\mathbb{R}$ or $\mathbb{C}$ and $M^n_\KK=\KK^{n\times n}$.

With $\Omega=\spmat{0&I_{n/2}\\-I_{n/2}}$ and $H=\spmat{I_p&0\\0&-I_q}$,
\begin{align}
 \mathop{\mathrm  {GL}}(n;\KK) &= \{g\in M^n_\KK\mathbin|\det g\neq 0\},&
 \mathop{\mathfrak{gl}}(n;\KK) &= \{a\in M^n_\KK\},\\
%
 \mathop{\mathrm  {SL}}(n;\KK) &= \{g\in M^n_\KK\mathbin|\det g=1\},&
 \mathop{\mathfrak{sl}}(n;\KK) &= \{a\in M^n_\KK\mathbin|\Tr a=0\},\\
%
 \mathop{\mathrm  {SU}}(n) &= \{g\in M^n_{\mathbb C}\mathbin|g^\dagger g=1\land\det g=1\},&
 \mathop{\mathfrak{su}}(n) &= \{a\in M^n_{\mathbb C}\mathbin|a+a^\dagger=0\land\Tr a=0\},\\
%
 \mathop{\mathrm  {SO}}(n) &= \{g\in M^n_{\mathbb R}\mathbin|g^\TT g=1\land\det g=1\},&
 \mathop{\mathfrak{so}}(n) &= \{a\in M^n_{\mathbb R}\mathbin|a+a^\TT=0\},\\
%
 \mathop{\mathrm  {SO}}(n;\mathbb C) &= \{g\in M^n_{\mathbb C}\mathbin|g^\TT g=1\land\det g=1\},&
 \mathop{\mathfrak{so}}(n;\mathbb C) &= \{a\in M^n_{\mathbb C}\mathbin|a+a^\TT=0\},\\
%
 \mathop{\mathrm  {SO}}(p,q) &= \{g\in M^{p+q}_{\mathbb R}\mathbin|H g^\TT H=g^{-1}\land\det g=1\},&
 \mathop{\mathfrak{so}}(p,q) &= \{a\in M^{p+q}_{\mathbb R}\mathbin|H a^\TT H=-a\},\\
%
 \mathop{\mathrm  {Sp}}(n;\KK) &= \{g\in M^n_\KK\mathbin|\Omega g^\TT \Omega=-g^{-1}\},&
 \mathop{\mathfrak{sp}}(n;\KK) &= \{g\in M^n_\KK\mathbin|\Omega a^\TT \Omega=a\},\\
%
 \mathop{\mathrm  {Sp}}(n) &= \{g\in M^n_{\mathbb C}\mathbin|\epsilon g^\TT \epsilon=-g^{-1}\land g^\dagger g=1\},&
 \mathop{\mathfrak{sp}}(n) &= \{g\in M^n_{\mathbb C}\mathbin|\epsilon a^\TT \epsilon=a\land a^\dagger+a=0\}.
\end{align}
Of course, $\mathop{\mathrm{U}}(n)$ and $\mathop{\mathrm{O}}(n;\KK)$ (and corresponding Lie algebra) are obtained by removing the condition $\det g=1$ (and $\Tr a=0$).


Important identities are:
\begin{equation}
 \gSp(2;\KK)=\gSL(2;\KK),\quad \gSp(2)=\gSU(2),
\end{equation}



The complexification of a finite-dimensional \emph{real} vector space $V$ is defined by
\begin{equation}
 V_{\mathbb C}=V\otimes\mathbb C=\{v_1+\ii v_2\mathbin|v_1,v_2\in V\}\cong V+\ii V.
\end{equation}
Then, complexification of a Lie algebra $\mathfrak a$ of a matrix Lie group is defined as follows.
Since $\mathfrak a$ is a \emph{real} Lie algebra, it can be complexified to $\mathfrak a_{\mathbb C}$, where its bracket operation has a unique extension so that $\mathfrak a_{\mathbb C}$ is a \emph{complex} lie algebra.
The complex Lie algebra $\mathfrak a_{\mathbb C}$ is called the complexification of $\mathfrak a$.
In particular,
\begin{align}\notag
 &
 \mathop{\mathfrak{gl}}(n;\mathbb R)_{\mathbb C}
 \cong \mathop{\mathfrak{u}}(n)_{\mathbb C}
 \cong \mathop{\mathfrak{gl}}(n;\mathbb C),&
 &
 \mathop{\mathfrak{so}}(n)_{\mathbb C}
 \cong \mathop{\mathfrak{so}}(n;\mathbb C),\\
 &
 \mathop{\mathfrak{sl}}(n;\mathbb R)_{\mathbb C}
 \cong \mathop{\mathfrak{su}}(n)_{\mathbb C}
 \cong \mathop{\mathfrak{sl}}(n;\mathbb C),&
 &
 \mathop{\mathfrak{sp}}(n;\mathbb R)_{\mathbb C}
 \cong \mathop{\mathfrak{sp}}(n)_{\mathbb C}
 \cong \mathop{\mathfrak{sp}}(n;\mathbb C).
\end{align}
On the other hand, a complex simple Lie algebra $\mathfrak a$ can be, on the other hand, viewed as a real Lie algebra, whose (real) dimension is twice as the original (complex) dimension.
We describe such real Lie algebra by $\mathfrak a_{\mathbb R}$.

Important identities are:
\begin{align}
 &\aSO(3,1)\cong \aSL(2;\mathbb C)_{\mathbb R},&
 &\aSO(3,1)_{\mathbb C}\cong \aSL(2;\mathbb C)\oplus\aSL(2;\mathbb C).
\end{align}


\subsection{Mathematical Foundations for Spinors}
\paragraph{Quadratic form}
Let~\cite[\S6.3]{Jacobson} $V$ be an $n$-dimensional vector space over a field $K$. A quadratic form $Q$ on $V$ is defined by
\begin{equation}
\begin{split}
   Q\colon V\to K,\qquad
   &\forall k\in K, v\in V, Q(k v)=k^2Q(v),\\
   &B\colon (v,w)\mapsto Q(v+w)-Q(v)-Q(w)\text{~is linear in $v$ and $w$.}
\end{split}
\end{equation}
The bilinear form $B$ satisfies $B(x,x)=2Q(x)$. As long as the characteristic of $F$ is not two, $Q$ is determined by $B$.

Equivalently, $Q$ is characterized by a symmetric ``matrix'' $\hat Q$ such that $Q(v)=v^\TT \hat Q v$.
Focusing on $K=\RR$, $\hat Q$ is a real symmetric matrix and thus diagonalizable. Sylvester's law of inertia states that there exists a basis such that $\hat Q=\diag(1,1,\cdots,1,-1,-1,\cdots,-1)$, where the number of $+1$ and $-1$ are respectively $p$ and $q$.
The pair $(p,q)$ is unique and is called the signature of $Q$, where $\rank \hat Q=p+q$.

Quadratic form induces orthogonality: a basis $\{e_i\}$ of $V$ is called orthogonal [orthonormal] under $Q$ if
\begin{equation}
  B(e_i,e_j)\propto \delta_{ij}\qquad\Bigl[B(e_i,e_j)=c_i\delta_{ij}\text{~with~}c_i\in\{2,0,-2\}\Bigr]
\end{equation}
and $(V,Q)$ has an orthogonal basis as long as the characteristic of $K$ is not two.
If $K$ is a ``spin field'' such as $\RR$ or $\CC$ (or, precisely, $\forall \alpha\in K,\exists \beta\in K$ s.t. $\alpha=\beta^2$ or $-\beta^2$), $(V,Q)$ has an orthonormal basis.

\paragraph{Geometric algebra}
Let~\cite{0907.5356} $V$ be an $n$-dimensional vector space over a field $K$. Let $Q$ be a quadratic form on $V$.
Tensor algebra on $V$ is defined by
\begin{equation}
  T(V) = \bigoplus_{k} V^{\otimes k}
       = K\oplus V\oplus (V\otimes V)\oplus\cdots.
\end{equation}
It has an ideal $K_Q$ generated by $\{v\otimes v - Q(v)\mid v\in V\}$:
\begin{equation}
  K_Q = \left\{\sum_{i} a_i\otimes(v_i\otimes v_i-Q(v_i))\otimes b_i\relmiddle| v_i\in V, a_i,b_i \in T(V)\right\}.
\end{equation}
Then the Geometric algebra $\GeoA(V,Q)$ is defined by $T(V)/K_Q$. This is isomorphic to $\Cl(E, K, q_E)$ defined below.

The product in $\GeoA(V,Q)$ is written by $vw \deq [v\otimes w]=v\otimes w+K_q$. Particularly, $v^2 = Q(v)$ and $vw+wv=B(v,w)$.

\paragraph{Clifford Algebra}
For a finite set $X$, a commutative ring $R$ with unit,
the Clifford algebra $\Cl(X,R,r)$ is defined as the free $R$-module generated by the set $2^X$ of all subset of $X$.
Here, $r\colon X\to R$ is an arbitrary function thought of as a signature.

The Clifford algebra $\mathrm{Cl}$ has
addition $\mathrm{Cl}\times\mathrm{Cl}\to\mathrm{Cl}$,
scalar multiplication $R\times\mathrm{Cl}\to\mathrm{Cl}$, and
product
\begin{equation}
  \mathrm{Cl}\times\mathrm{Cl}\to\mathrm{Cl}\ni AB
  \deq \tau(A,B)A\setdiff B,
\end{equation}
where $A\setdiff B=(A\cup B)\setminus(B\cap A)$,
defined based on the following map $\tau\colon2^X\times2^X\to R$:
\begin{equation}
  \begin{split}
  &\tau(\{x\},\{x\}) = r(x)\quad\forall x\in X,&
  &\tau(A,B)\tau(A\setdiff B,C)=\tau(A,B\setdiff C)\tau(B,C)\quad\forall A,B,C\in 2^X,\\
  &\tau(\{y\},\{x\}) = -\tau(\{x\},\{y\})\quad\forall x,y\in X,x\neq y,&
  &\tau(\emptyset,A)=\tau(A,\emptyset)=1\quad\forall A\in2^X,\\
  &\tau(A,B)\in\{-1,1\}\quad \text{if}~A\cap B=\emptyset.
  \end{split}
\end{equation}

Several standard operations are defined: for $A\in2^X$ and $k=\binom{|A|}{2}$,
\begin{equation}
\text{Grade involution}\quad A^\star\deq(-1)^{|A|}A,\qquad
\text{Reversion}\quad A^\dagger\deq(-1)^{k}A,\qquad
\text{Clifford conjugate}\quad A^\mdlgwhtsquare\deq A^{\star\dagger}.
\end{equation}
The sign of grade involution (reversion) is negative iff $|A|$ mod 4 is 1 or 3 (2 or 3).


Let $(V,Q)$ be an $n$-dimensional vector space $V$ over a field $K$ equipped with a quadratic form $Q$. Let $E=\{e_i\}$ be an orthogonal basis of $(V,Q)$.
Then
\begin{equation}
  \Cl(E, K, Q|_E) \cong \GeoA(V,Q).
\end{equation}
The Clifford algebra $\Cl(X,R,r)$ has subspaces of $k$-vectors $\mathrm{Cl}^k$ and of even and odd vectors $\mathrm{Cl}^\pm$:
\[
\Cl(X,R,r)=\mathrm{Cl}^+\oplus\mathrm{Cl}^-=
\mathrm{Cl}^0\oplus\mathrm{Cl}^1\oplus\mathrm{Cl}^2\oplus\cdots\oplus\mathrm{Cl}^{|X|}.
\]
In particular, $\mathrm{Cl}^+$ is a subalgebra.

\paragraph{Groups in Geometric Algebra}
The following groups are embedded in $\GeoA(V,Q)$,
where $V^\times=\{v\in V\mid Q(v)\neq 0\}$:
\begin{align}
  &\GeoA^\times \deq \{g\in\GeoA\mid \exists g^{-1}\in\GeoA, gg^{-1}=g^{-1}g=1\},
  \qquad \Gamma\deq\{v_1v_2\cdots v_n\mid v_i\in V^\times\},
  \qquad \tilde \Gamma\deq \{g\in\GeoA^\times\mid gVg^{-1}\in V\},
\notag\\
  &\Pin\deq\{x\in \Gamma\mid xx^\dagger = \pm1\},
 \qquad\Spin\deq \Pin\cap\GeoA^+,
 \qquad\Spin^+\deq \{x\in\Spin\mid xx^\dagger=1\},
\end{align}
The Lipschitz group $\tilde\Gamma$ is actually equal to the versor group $\Gamma$ and is the smallest group that contains $V^\times$. To summarize,
\[\GeoA\supset\GeoA^\times\supset \Gamma=\tilde\Gamma\supset V^\times;\qquad\Gamma\supset\Pin\supset\Spin\supset\Spin^+.\]

There are surjective homomorphisms with kernel $\pm1$ for
\begin{equation}
  \Pin(s,t)\to\gO(s,t),\qquad
  \Spin(s,t)\to\gSO(s,t),\qquad
  \Spin^+(s,t)\to\gSO^+(s,t).
\end{equation}
Also, for our interest, there are isomorphisms~\cite{RauschdeTraubenberg:2005aa,Yamaguchi:spinor}
\begin{equation}
  \GeoA(\RR^{4,1})\cong \GeoA(\RR^{1,3})\otimes \CC \cong \GeoA(\CC^4)\cong\CC^{4\times4},\qquad
  \Spin^+(1,3)\cong\gSL(2,\CC)\cong\gSp(2,\CC),\qquad
  L_0\cong\mathop{\mathrm{PSL}}(2;\CC)=\gSL(2;\CC)/\ZZ_2.
\end{equation}
Meanwhile, the Lorentz algebra $\aSO(1,3)$ is isomorphic to $\aSL(2;\mathbb C)$ viewed as a real Lie albegra~\cite[\S7.8]{Hall2015},
and its complexification
$\aSO(1,3)_{\mathbb C}$ is isomorphic to $
\aSU(2)_{\mathbb C}\oplus\aSU(2)_{\mathbb C}
=\aSL(2,\mathbb C)\oplus\aSL(2,\mathbb C)$.



\paragraph{Clifford algebra and spin group}\TODO{needs experts...}
We just provide a little information on spin group $\mathop{\mathrm Spin}(1,3)$ associated to (our) Minkowski metric $\eta=\diag(+,-,-,-)$:~\cite{RauschdeTraubenberg:2005aa,Yamaguchi:spinor,Wikipedia:Clifford_Algebra}. To define the Clifford algebra $\mathfrak C(V,Q)$, let
\begin{itemize}
 \item[$V$]: a vector space over a field $K$; in our case, $V=\mathbb R^{1,3}$ and $K=\mathbb R$.
 \item[$Q$]: a quadratic form on $V$,
 \item[$A$]: %
  a unital associative algebra over $K$; it means that $(A, +, \ast)$ is a ring and is also a vector space on $K$, it should satisfy
  $(k a_1)\ast a_2=a_1\ast(ka_2)=k(a_1\ast a_2)$ for $a_i\in A$ and $k\in K$, and it should contain $1_A$ such that $1_A\ast a=a\ast 1_A=a$;
  here a map $f:K\to A$ with $f(k)=k1_A$ is an injective ring homomorphism, with which $K$ can be included in $A$.
\end{itemize}
Then we can define a linear map $i:V\to \mathfrak C(V,Q)$ satisfying $i(v)\ast i(v)=Q(v)1_A$ for all $v\in V$ as follows, which defines the Clifford algebra.
Given a linear map $j:V\to A$ satisfying $j(v)\ast j(v)=Q(v)1_A$ for all $v\in V$, there is a unique algebra homomorphism $f:\mathfrak C(V,Q)\to A$ such that $f\circ i=j$.

We focus on $\mathbb R^{1,3}$ and $Q(v)=\eta_{\mu\nu}v^\mu v^\nu$ (Minkowski norm). Then, by introducing the basis $e_\mu$ of $V$, the condition yields
\begin{align}
& (v^2+2v^\mu w_\mu + w^2)1_A = [j(v)+j(w)]\ast[j(v)+j(w)] = (v^2+w^2)1_A+j(v)\ast j(w)+j(w)\ast j(v),\notag\\
&\therefore \{j(v),j(w)\}=2v^\mu w_\mu 1_A
\text{~~and thus~~} \{e_\mu,e_\nu\}=2\eta_{\mu\nu} 1_A.\label{eq:ClifAl}
\end{align}
The Clifford algebra $\mathfrak C(\mathbb R^{1,3},\eta)$ is thus seen as the associative algebra generated by $\{1_A, e_0, e_1, e_2, e_3\}$ satisfying \cref{eq:ClifAl}.

Several properties of the Clifford algebra $\mathfrak C_{1,3}$ is in order:
\begin{itemize}
 \item It is of dimension $2^4=16$ and the basis is given by
$\{1, e_\mu, (e_\mu e_\nu)_{\mu<\nu}, (e_\mu e_\nu e_\rho)_{\mu<\nu<\rho}, e_0e_1e_2e_3\}.$
 \item It contains a subalgebra $\mathfrak C^0_{1,3}$ spanned by $\{1, e_\mu e_\nu, e_0e_1e_2e_3\}$, while the remaining vector space $\mathfrak C^1_{1,3}$ is by $\{e_\mu, e_\mu e_\nu e_\rho\}$.
 \item We define $\epsilon=e_0e_1e_2e_3$, which satisfies $\epsilon^2=-1$ and $[\epsilon,\mathfrak C^0_{1,3}]=\{\epsilon,\mathfrak C^1_{1,3}\}=0$.
 \item $\aSO(1,3)\subset\mathfrak C_{1,3}$; explicitly, the generators are given by $S_{\mu\nu}:={[e_\mu,e_\nu]}/{4}$, which satisfy
\begin{equation}
 [S_{\mu\nu},e_\rho] = e_\mu\eta_{\nu\rho}-e_\nu\eta_{\mu\rho},\qquad
 [S_{\mu\nu},S_{\rho\sigma}] = 
-\eta_{\mu\rho}S_{\nu\sigma}
+\eta_{\nu\rho}S_{\mu\sigma}
+\eta_{\mu\sigma}S_{\nu\rho}
-\eta_{\nu\sigma}S_{\mu\rho}.
\end{equation}
\end{itemize}

The Clifford group is defined by
\begin{equation}
\Gamma_{1,3}=\{x\in\mathfrak C_{1,3}\mathbin|\forall v\in \mathbb R^{1,3}, xvx^{-1}\in \mathbb R^{1,3}\},
\end{equation}
i.e., $x$ should return a ``vector'' $v^\nu e_\nu\in \mathbb R^{1,3}$ into another ``vector'', which in fact corresponds to the Lorentz transformation:
$xe_\nu x^{-1} = (\Lambda^{-1})_\nu{}^\mu e_\mu$.
The correspondence is however one-to-$N$, and thus it should be normalized by the squared-norm $N(x)$
\begin{equation}
 N(x)=x\overline{x};\qquad\overline{e_\mu}=e_\mu,\quad\overline{ke_\mu e_\nu e_\rho}=ke_\rho e_\nu e_\mu, \text{etc., where $k\in\mathbb R$}.
\end{equation}
Then, we define
\begin{align*}
 \mathop{\mathrm{Pin}}(1,3)&=\{x\in\Gamma_{1,3}\mathbin||N(x)|=1\},&
 \mathop{\mathrm{Spin}}(1,3)&=\mathop{\mathrm{Pin}}(1,3)\cap \mathfrak C^0_{1,3},&
 \mathop{\mathrm{Spin}}(1,3)^+&=\{x\in\mathop{\mathrm{Spin}}(1,3)\mathbin|N(x)=1\},
\end{align*}
whose properties are in order:
\begin{itemize}
 \item $\mathop{\mathrm{[S]pin}}(1,3)$ is a double covering of $\mathop{\mathrm{[S]O}}(1,3)$ but it is not connected.
 \item $\mathop{\mathrm{Spin}}(1,3)^+$ is a connected Lie group and is a double covering of $\gSO(1,3)^+$.
 \item $\mathop{\mathrm{Spin}}(1,3)^+\cong\gSL(2;\mathbb C)\cong\gSp(2;\mathbb C)$.
\end{itemize}

\end{document}
