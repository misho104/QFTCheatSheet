\documentclass[CheatSheet]{subfiles}
\begin{document}
\summarystyle

\section{Gauge theory}
\paragraph{SU(N)}Fundamental rep.~$\boldsymbol N\sim (\tau^{a})_{ij}$ (Hermitian), $\overline{\boldsymbol N}\sim(-\tau^{a*})_{ij}$, and adjoint rep.~$\text{\textbf{adj.}}\sim (f^a)^{bc}$.\footnote{Upper and lower gauge indices are equivalent, while Lorentz indices and Weyl-spinor indices are different for super- and subscripts because they are raised/lowered by, e.g., metric tensors.}
\begin{align*}
 &\Tr(\tau_a\tau_b)=\frac12\delta_{ab},\qquad
 [\tau_a,\tau_b]=\ii f_{abc}\tau_c,&
 &[\tau_a,[\tau_b,\tau_c]] = [[\tau_a,\tau_b],\tau_c] + [\tau_b,[\tau_a,\tau_c]],\\
 &f^{abc}=-2\ii\Tr([\tau^a,\tau^b]\tau^c): \text{real, anti-symmetric},&
 &f^{ade}f^{bcd} + f^{bde}f^{cad} + f^{cde}f^{abd} = 0.
\end{align*}
\begin{align}
 \boldsymbol{N}_i
&\mapsto [\exp(\ii g\theta^a \tau^a)]_{ij}\boldsymbol{N}_j
\simeq  \boldsymbol{N}_i + \ii g\theta^a \tau^a_{ij}\boldsymbol{N}_j\\\notag
 \overline{\boldsymbol{N}}_i
&\mapsto
\overline{\boldsymbol{N}}_j [\exp(-\ii g\theta^a \tau^a)]_{ji}
= [\exp(-\ii g\theta^a \tau^{a*})]_{ij} \overline{\boldsymbol{N}}_j \qquad\text{(i.e., $\overline{\boldsymbol{N}}_i=\boldsymbol N_i^*$)}\\
&\simeq\overline{\boldsymbol{N}}_j -\ii g\theta^a \overline{\boldsymbol{N}}_j\tau^a_{ji}
\simeq \overline{\boldsymbol{N}}_j -\ii g\theta^a \tau^{a*}_{ij} \overline{\boldsymbol{N}}_j\end{align}

\paragraph{SU(2)} Fundamental representation $\TWO\sim T^a \equiv \sigma^a/2$ and adjoint representation $\THREE\sim \nep^{abc}$.
\begin{equation*}
 \Tr(T_aT_b)=\frac12\delta_{ab},\qquad
 [T_a,T_b]=\ii \nep_{abc}T_c,\qquad
 \overline\TWO = \TWO^* = \nep\TWO\quad (\because T^*=\nep T \nep=-\nep T\nep^{-1}),
\end{equation*}
where the last identity comes as follows:
\begin{equation}
\nep_{ij} \TWO_j
\mapsto
\nep_{ij}\left(
[\exp(\ii g\theta^a T^a)]_{jk}\TWO_k
\right)
= [\nep\exp(\ii g\theta^a T^a)\nep^{-1}]_{ij}\nep\TWO_j
= [\exp(-\ii g\theta^a T^{a*})]_{ij}(\nep_{jk}\TWO_k).
\end{equation}

\paragraph{SU(3)} Fundamental rep.~$\THREE\sim\tau^a\equiv \lambda^a/2$, $\overline\THREE\sim(-\tau^{a*})$, and adjoint rep.~$\boldsymbol8\sim (f^a)^{bc}$.
\begin{align}\notag
\THREE:~\phi_a&
  \to [\exp(\ii g \theta^\alpha \tau^\alpha)]_{ab}\phi_b,
&
\THREEbar:~\phi_a&
  \to[\exp(-\ii g \theta^\alpha \tau^{\alpha*})]_{ab}\phi_b,
\\
\phi_a^*&
  \to[\exp(-\ii g \theta^\alpha \tau^{\alpha*})]_{ab}\phi_b^*,
&
\phi_a^*&
  \to[\exp(\ii g \theta^\alpha \tau^{\alpha})]_{ab}\phi_b^*.
\end{align}

\clearpage
\detailstyle

\subsection{Gell-Mann matrices}
Gell-Mann matrices and a Mathematica code to generate them are:
\begin{align}
 \lambda_{1\text{--}8} &=
  \left(\begin{smallmatrix} 0&1&0 \\ 1&0&0 \\ 0&0&0 \end{smallmatrix}\right),
  \left(\begin{smallmatrix} 0&-\ii&0 \\ \ii&0&0 \\ 0&0&0  \end{smallmatrix}\right),
  \left(\begin{smallmatrix} 1&0&0 \\ 0&-1&0 \\ 0&0&0  \end{smallmatrix}\right),
  \left(\begin{smallmatrix} 0&0&1 \\ 0&0&0 \\ 1&0&0  \end{smallmatrix}\right),
  \left(\begin{smallmatrix} 0&0&-\ii \\ 0&0&0 \\ \ii&0&0  \end{smallmatrix}\right),
  \left(\begin{smallmatrix} 0&0&0 \\ 0&0&1 \\ 0&1&0  \end{smallmatrix}\right),
  \left(\begin{smallmatrix} 0&0&0 \\ 0&0&-\ii \\ 0&\ii&0  \end{smallmatrix}\right),
  \tfrac1{\sqrt3}\left(\begin{smallmatrix} 1&0&0 \\ 0&1&0 \\ 0&0&-2 \end{smallmatrix}\right).
\end{align}
\begin{minted}{wolfram}
GellMann[0] := DiagonalMatrix[{1,1,1}]/Sqrt[3/2]
GellMann[8] := DiagonalMatrix[{1,1,-2}]/Sqrt[3]
GellMann[a:1|2|3|4|5|6|7] := Module[
   {p=Switch[a,1|2|3,{1,2,0},4|5,{1,0,2},6|7,{0,1,2}]},
   Table[If[i*j==0, 0, PauliMatrix[{1,2,3,1,2,1,2}[[a]]][[i,j]]], {i,p}, {j,p}]]
\end{minted}


\end{document}
